% Created 2021-11-17 Wed 12:14
% Intended LaTeX compiler: pdflatex
\documentclass{dmghandout}
\author{EDT}
\date{11/18/2021}
\title{The COMPAS Debate}
\hypersetup{
 pdfauthor={EDT},
 pdftitle={The COMPAS Debate},
 pdfkeywords={},
 pdfsubject={},
 pdfcreator={Emacs 27.2 (Org mode 9.5)}, 
 pdflang={English}}
\begin{document}


\section*{COMPAS}
\label{sec:org6fc921d}

Recidivism prediction instrument developed by Northpointe, inc.
\begin{itemize}
\item Estimates risk of committing a crime within 2 years
\item Uses probability of being charged with a new crime as a proxy
\end{itemize}

\section*{TP, TN, FP, FN}
\label{sec:orga06fa52}

\emph{True positive (TP):} labeled high risk, reoffends \\
\emph{True negative (TN):} labeled low risk, doesn't reoffend \\
\emph{False positive (FP):} labeled high risk, doesn't reoffend \\
\emph{False negative (FN):} labeled low risk, reoffends \\

\section*{False positive rate}
\label{sec:org470674d}

The probability that a randomly selected defendant who does not reoffend will be labeled “high risk”

$$\text{FPR} = \frac{\text{FP}}{\text{FP+TN}} = \frac{\text{labeled HR, no recidivism}}{\text{no recidivism}}$$

\section*{Equal False Positive Rates}
\label{sec:orgd7155ae}

The false positive rate for black defendants is similar to the false positive rate for white defendants

\section*{Equal Calibration}
\label{sec:org1c877f6}

Defendants who receive the same risk score reoffend at similar rates, regardless of race

\section*{Base Rate}
\label{sec:org5c219c1}

The percentage of defendants from the group that reoffend

$$\text{BR} = \frac{\text{TP+FN}}{\text{TP+TN+FP+FN}} = \frac{\text{defendants that reoffend}}{\text{all defendants}}$$
\end{document}
